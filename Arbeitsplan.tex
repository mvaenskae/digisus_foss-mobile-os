\documentclass{article}


\begin{document}

\section{Arbeitsauftrag}

Aus dem Forum:

1. Neben Android etablieren sich (langsam) andere Smartphone-Betriebssysteme, die ebenfalls offen sind. Definiert zum Einstieg, was ein offenes im Vergleich zu einem geschlossenen mobilen Betriebssystem ist.

2. Wählt sechs Systeme aus, die ihr in Bezug auf technische und rechtliche Offenheit sowie Community/Ökosystem analysiert und vergleicht sie. Mindestens dabei sein sollten: Android (evtl. mit Variante Cyanogenmod), Firefox OS, Sailfish OS, …

3. Was bedeutet es für die Hardware-Anbieter und die Endkunden, wenn offene Smartphones zur Verfügung stehen? Wie, mit welchen Argumenten, erklärt ihr einem iPhone Benutzer den Unterschied?

\section{Ziele}

Wir m\"{o}chten:

\begin{itemize}
    \item Definieren, was ein offenes mobiles OS ausmacht
    \item verschiedene offene OS miteinander und mit geschlossenen Varianten vergleichen
    \item Herausfinden, was f\"{u}r Probleme offene mobile OS auf dem Markt haben und warum
    \item Die Bedeutung der offenen mobilen OS f\"{u}r den Markt untersuchen
    \item Parallelen und Unterschiede zur PC-Welt beleuchten
\end{itemize}

\section{Inhaltsverzeichnis}

Dieses Inhaltsverzeichnis ist provisorisch. Wir nehmen uns die Freiheit, es gegebenenfalls anzupassen.

\begin{itemize}
    \item Einf\"{u}hrung
    \begin{itemize}
        \item Motivation
        \item Definition 'offees mobiles Betriebssystem'
        \item technische vs. rechtliche Offenheit
        \item Unterscheidung
    \end{itemize}
    \item Entwicklung
    \begin{itemize}
        \item Vergleich der Entwicklung von Smartphones und PCs
        \item Erkl\"{a}rungsversuch Unterschiede Smartphones/PCs
    \end{itemize}
    \item Analyse verschiedener Betriebssysteme
    \begin{itemize}
        \item Android (Aline)
		\begin{itemize}
			\item Einf\"{u}hrung
			\item technische Offenheit
			\item rechtliche Offenheit
			\item \"{O}kosystem/Community
			\item ggf. Weiteres...
		\end{itemize}
        \item Firefox OS (Aline)
        \item Sailfish OS (Mickey)
        \item Maemo (Mickey)
        \item Ubuntu Phone (Sandro)
        \item iOS (Sandro)
    \end{itemize}
    \item Auswirkungen
    \begin{itemize}
        \item Auf die Hardware-Anbieter
        \item Auf den Markt
    \end{itemize}
    \item Fazit
    \begin{itemize}
        \item Stand mobiler offener OS heute
        \item Zukunft
    \end{itemize}
\end{itemize}



\section{Arbeitsplan}
Diesen m\"{o}chten wir erst erstellen, wenn die Deadlines bekannt sind.

\end{document}
