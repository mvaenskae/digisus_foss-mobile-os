\begin{titlepage}

\thispagestyle{plain}
\fancyhead{}
\fancyfoot{}
\fancyhead[R]{\includegraphics[width=5.5cm]{eth.png}}

\begin{center}
\vspace{3.5cm}
\rule{\linewidth}{0.5mm} \\[0.4cm]
\textbf{\huge G03 - Open Source Mobiles} \\
\rule{\linewidth}{0.5mm} \\
\vspace{3\baselineskip}
\end{center}
\begin{large}
\begin{enumerate}
    \item Neben Android etablieren sich (langsam) andere mobile Smartphone-Betriebssystem, die ebenfalls offen sind. Definiert zum Einstieg, was ein offenes im Vergleich zu einem geschlossenen mobilen Betriebssystem ist.

    \item Wählt sechs Systeme aus, die ihr in Bezug auf technische und rechtliche Offenheit sowie Community/Ökosystem analysiert und vergleicht sie. Mindestens dabei sein sollten: Android (evtl. mit Variante Cyanogenmod), Firefox OS, Sailfish OS, …

    \item Was bedeutet es für die Hardware-Anbieter und die Endkunden, wenn offene Smartphones zur Verfügung stehen? Wie, mit welchen Argumenten, erklärt ihr einem iPhone Benutzer den Unterschied?
\end{enumerate}

\begin{center}
\vfill

 Aline Abler, \textsc{D-INFK}, \emph{ablera@student.ethz.ch} \\
Sandro Kalbermatter, \textsc{D-INFK}, \emph{sandroka@student.ethz.ch} \\
Mickey Vänskä, \textsc{D-INFK}, \emph{mickeyv@student.ethz.ch} \\
\vspace{2\baselineskip}

Dieser Bericht entstand im Rahmen der Vorlesung \mbox{``Digitale Nachhaltigkeit in der Wissensgesellschaft''} bei Dr{.} Marcus M{.} Dapp \\
\vspace{2\baselineskip}

Er darf gemäss folgender Creative Commons Lizens verwendet werden: \mbox{CC BY-NC-SA 3.0} \url{https://creativecommons.org/licenses/by-nc-sa/3.0/legalcode} \\
\vfill

Herbstsemeter 2015 \\
\vspace{0.5\baselineskip}
ETH Zürich

\end{center}
\end{large}

\end{titlepage}
