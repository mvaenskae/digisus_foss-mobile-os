iOS, früher iPhone OS, ist das mobile Betriebssystem von Apple und kommt auf allen iPhones und iPads zum Einsatz. Es wird ausschliesslich für Hardware von Apple entwickelt\thinspace\cite{online:ios-wikipedia}. iOS gilt als das erste mobile Betriebssystem, das im grossen Stil bekannt wurde und setzte sich rasch international durch.

iOS war von Anfang an in Apples Hand und wurde von Steve Jobs selbst in Auftrag gegeben, wodurch es sich von vielen in diesem Dokument diskutierten Systemen unterscheidet\thinspace\cite{online:ios-wikipedia}. Der damals bereits weltweit etablierte Computergigant verfügte über das nötige \mbox{Know-How} und die Finanzkraft für die Entwicklung dieses mobilen Betriebssystems. Ziel war es, ein kommerzielles Produkt zu schaffen, welches zusammen mit den damit ausgelieferten Geräten marktfähig war.

Apps wurden zu Beginn noch aus dem iTunes Store geladen\thinspace\cite{online:ios-appstore}. Mit dem iPhone 3G wurde der AppStore eingeführt, der bis heute als Hauptquelle für den Bezug von Apps zu nutzen ist. Wie alle Komponenten des Systems ist auch dieser Store fest in Apples Hand.
\newline

\subsection{So geschlossen wie technisch machbar}
Aus eigener Erfahrung ist es ein wichtiger Aspekt in Apples Sicherheitsrichtlinie, dem Nutzer nur die Rechte zu geben, die er absolut benötigt. Dies macht sich auch in iOS bemerkbar: Es ist nur möglich, im Rahmen des vom Hersteller Vorgesehenen zu arbeiten.

Apps können offiziell nur aus Apples AppStore bezogen werden. Für Firmen gibt es noch eine zweite Lösung: die sogenannte ``Enterprise/Corporate development license''. Ein Anwendungsbeispiel dafür ist\thinspace\cite{online:ios-xamarin}. Diese wird jedoch ebenfalls von Apple kontrolliert\thinspace\cite{online:ios-appsfromoutside}. Apps, die nicht von Apple genehmigt wurden, können offiziell nicht installiert werden\thinspace\cite{online:ios-appstore}.

Eine Nutzung von iOS ist heutzutage ohne ein Konto bei Apple unrealistisch. Nutzer haben sich zu authentifizieren, um Backups ihrer Daten zu erstellen, Apps zu laden oder Kontakte zu synchronisieren. iTunes ist offiziell der einzige Weg zur Verwaltung des mobilen Geräts von einem Computer aus.

Apple arbeitet hart daran, den Benutzer daran zu hindern, Veränderungen am System vorzunehmen, welche über die vorgegebenen Einstellung hinaus gehen. Dennoch gelingt es der Community immer wieder, innert weniger Wochen oder Monaten Wege zu finden, diese Vorkehrungen zu umgehen. Das ``Befreien'' des Systems aus dem geschützten Zustand wird ``Jailbreak'' genannt\thinspace\cite{online:ios-jailbreak}. Dabei geht jedoch in gewissen Ländern die Herstellergarantie verloren und es entstehen Sicherheitslücken. Auch besteht die Gefahr, das Gerät unbrauchbar zu machen (zu "bricken")\thinspace\cite{online:ios-whyjailbreak}. Apple nennt Jailbreaking ``unerlaubte Modifikationen'', rät explizit davon ab und weist darauf hin, dass das ``Hacken'' von iOS gegen die EULA verstösst\thinspace\cite{online:ios-dontjailbreak}.
\newline

\subsection{All rights reserved, GPL-Apps rechtlich nicht möglich}
Was die rechtliche Offenheit anbelangt, ist die Situation einfach: Sämtliche Komponenten sind von Apple entwickelt, alle Rechte vorbehalten. Quellcode ist nicht öffentlich einsehbar. Insofern ist iOS so geschlossen wie möglich.

Auch für \mbox{App-Entwickler} gelten strikte Bedingungen: um ihre Software offiziell für iOS anbieten zu können, müssen Entwickler mindestens \$99 pro Jahr an Apple entrichten\thinspace\cite{online:ios-appstore}. Des Weiteren gibt es viele Einschränkungen für iOS Apps --- darunter die, dass Mobile Betriebssysteme von Drittanbietern nicht im Wortlaut erwähnt werden dürfen\thinspace\cite{online:ios-guidelines}. Die wohl folgenschwerste dieser Einschränkungen ist die Klausel, die den Entwickler zwingt, die dem Nutzer gewährten Rechte im Sinne von Apple einzuschränken\thinspace\cite{online:ios-terms}. Dies ist ein Widerspruch zur GPL, weswegen es Entwicklern nicht möglich ist, \mbox{GPL-lizenzierte} Software für ihre iOS Apps zu verwenden\thinspace\cite{online:ios-gpl}.
\newline

\subsection{Die Apple-Community}
Alle Apple-Geräte, darunter Mac, iPhone und iPad, werden von Apple und nur von Apple entwickelt. Dies erlaubt es dem Unternehmen, seine eigene Biosphäre zu schaffen, die den Kunden sowohl im mobilen als auch im stationären Feld umgibt. Apple harmonisiert iOS mit Mac OS: Seit OS\thinspace X Yosemite sind iOS und Mac OS fest verknüpft und synchronisieren diverse Daten über verschiedene \mbox{Apple-Geräte} hinweg. Dies führt dazu, dass die \mbox{Mac-Community} meist auch iOS nutzt. Die Gegenrichtung ist nicht immer der Fall, wohl auch wegen den hohen Preisen im Vergleich zur Konkurrenz.

Die \mbox{Apple-Community} beherbergt viele sehr überzeugte Fans, die beim Erscheinen eines neuen Geräts bekanntermassen sogar die Nacht vor dem Apple Shop verbringen. Zwischen Nutzern von iOS und anderen mobilen Betriebssystemen herrscht eine fast schon religiös anmutende Kluft: \mbox{Apple-Nutzer} werden teils gar als ``Hipster''\thinspace\cite{online:ios-hipster} bezeichnet und verstehen ihrerseits oft nicht, wie jemand irgend ein anderes System benutzen könnte. Dieses soziale Phänomen prägt die gesamte Apple-Community und ist in dem Ausmass in der Branche einzigartig.