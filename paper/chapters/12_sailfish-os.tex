Sailfish OS (auch \mbox{SailfishOS}) wurde 2011 von Jolla gestartet und erstmals Ende November 2013\thinspace\cite{online:jolla-about} veröffentlicht. Es wird hauptsächlich von Jolla entwickelt und basiert auf dem Linux Kernel, dem Mer Projekt, welches Basisfunktionalitäten anbietet, einer eigenen UI, sowie standardmäßigen Apps für die nötigsten Funktionalitäten\thinspace\cite{online:sailfish-about}. Der Linux Kernel ist GPL v2 lizenziert\thinspace\cite{online:kernel-license}, das Mer Projekt unter verschiedenen FOSS Lizenzen\thinspace\cite{online:mer-license}, die UI ist proprietär und die mitgelieferten Applikationen sind eine Mischung\thinspace\cite{online:sailfish-about}, wobei eine private, \mbox{nicht-kommerzielle} Nutzung des Systems seitens Jolla erlaubt ist\thinspace\cite{online:sailfish-eula}. \mbox{SailfishOS} wird in 2 Versionen ausgeliefert, eine mit proprietären Komponenten und eine mit nur freier Software. Jolla liefert mit ihrer Hardware des weiteren noch die proprietären Erweiterungen\thinspace\cite{online:sailfish-about} ``Android runtime''\thinspace\cite{online:sailfish-android-runtime-licensor}, ``Exchange active sync''\thinspace\cite{online:microsoft-eas} und ``Text prediction engine''\thinspace\cite{online:jolla-xt9}\thinspace\cite{online:xt9-license} aus. Erstere emuliert eine \mbox{Android-Umgebung}, welche es den Benutzern erlaubt, Android-Anwendungen mit gewissen Beschränkungen auszuführen\thinspace\cite{online:sailfish-android-runtime}. Letztere sind direkt oder indirekt mit Patenten versetzt.
\newline

\subsection{Das System}
Im Auslieferungszustand bekommen User die nötigsten Funktionalitäten vom System. Leider sind nur der Webbrowser und die \mbox{Office-Suite} unter einer freien Lizenz\thinspace\cite{online:sailfish-about}. Einen Kontrast zu diesen restriktiven Apps bildet das Erlangen von \mbox{``root''-Rechten}, welches gut dokumentiert und schnell durchführbar ist\thinspace\cite{online:sailfish-root}. Die User-Interaktion erfolgt bei \mbox{SailfishOS} mittels Gesten; Buttons sucht man vergeblich. Der Grundgedanke dessen ist es, eine Umgebung zu schaffen, welche dem User ein schickes und elegantes Interface bietet, das den vollen Bildschirm ausnutzt\thinspace\cite{online:sailfish-ui}. Im Moment ist dies noch ein Grundbestandteil des Systems, es soll aber in Zukunft entfernt werden\thinspace\cite{online:sailfish-about}. Das System würde in diesem Fall freier werden.

Applikationen, welche nativ ausgeführt werden, können aus zwei App-Stores (``\mbox{Jolla-Store}'' und ``\mbox{OpenRepos}''\thinspace\cite{online:openrepos}) installiert werden, wobei das Betreiben eines eigenen AppStores technisch möglich ist. Die Applikationen werden im ``\mbox{.rpm}'' Format ausgeliefert, welches frei und offen ist\thinspace\cite{online:rpm-license}. Ein offenes Format erlaubt es den Nutzern zu überprüfen, was das Paket am System bei der Installation verändert. Dies stärkt  die Freiheit der Nutzer. Leider kann der \mbox{Jolla-Store} nicht ohne entsprechenden Client besucht werden und es ist zwingend ein Jolla-Account nötig\thinspace\cite{online:jolla-store}. Der Client für \mbox{OpenRepos} scheint nicht Open Source zu sein, und der entsprechende AppStore hat weder eine Nutzungsbedingung, noch werden Lizenzen explizit erwähnt.

Anzumerken ist die Existenz von GCC im Auslieferungszustand. Jenes ermöglicht Nutzern selbstgeschriebene Software direkt auf dem System zu kompilieren\thinspace\cite{online:sailfish-list-licenses} und auszuführen.

Applikationen, welche emuliert ausgeführt werden (auf Android-Basis), haben die Lizenzierungen des entsprechenden ``apk''s und AppStores.
\newline

\subsection{Community}
Die Community setzt sich zu einem Großteil aus technikerfahreneren Leuten zusammen und solchen, die ein System benutzen wollen, welches offener ist als die drei aktuell größten Anbieter. Neben einer Mailingliste gibt es ein Forum und mehrere \mbox{IRC-Channels}\thinspace\cite{online:sailfish-communitygeneral}. Dazu gibt es immer wieder Treffen in \mbox{IRC-Channels}, bei welchen vorgeschlagene Themen angesprochen werden. Des Weiteren nehmen \mbox{SailfishOS} Entwickler an Konferenzen teil und zeigen dort ihre Fortschritte oder anderweitige Neuerungen am System.
\newline

\subsection{Hardware Support}
\mbox{SailfishOS} kann auf vielen verschiedenen Endgeräten installiert werden. Neben dem Raspberry Pi 2\thinspace\cite{online:sailfish-rpi2} wurde \mbox{SailfishOS} von der Community\thinspace\cite{online:sailfish-android-port} auf einige Android-Geräte portiert. Diese Ports sind je nach Motivation und Anzahl der Entwickler mehr oder weniger ausgereift\thinspace\cite{online:sailfish-porters}.

Jolla ist bereit ihre Version an OEMs zu lizenzieren. Um Interesse zu wecken, entwickelte Jolla bisher Hardware --- Handy\thinspace\cite{online:jolla-smartphone} und Tablet\thinspace\cite{online:jolla-tablet} ---, ist aber im Prozess der Restrukturierung in eine Software-Firma und einen Hardwarehersteller\thinspace\cite{online:jolla-googlelike}. Es lassen sich starke Parallelen zu Google ziehen und deren Umgang mit Android sowie der Nexus-Linie.

Jollas Plan trägt bereits erste Früchte. Es haben sich zwei Hardware-Hersteller gefunden, welche \mbox{SailfishOS} auf ihren Geräten anbieten möchten. Zum einen Fairphone auf ihrem Fairphone 2\thinspace\cite{online:jolla-fairphone2} und zum anderen Intex Technologies (zweitgrößter Smartphone Hersteller indischen Ursprungs), die Ende 2015 in Indien erste Hardware auf \mbox{SailfishOS} Basis anbieten wollen\thinspace\cite{online:jolla-intex-pdf}. In beiden Fällen wird Jollas proprietäre \mbox{SailfishOS} Version ausgeliefert.
\newline

\subsection{Fazit}
\mbox{SailfishOS} erfüllt mit seiner \mbox{``root''-Unterstützung}, dem \mbox{rpm-Format} und dem Erhalt der Garantie sowie Funktionalität unsere Aspekte. Jedoch ist es durch seine proprietären Komponenten bei den \mbox{Standard-Apps} und der \mbox{System-UI} nicht vollkommen offen. Die \mbox{FOSS-Version} von \mbox{SailfishOS} ist leider nicht für den alltäglichen Gebrauch geeignet, da UI und \mbox{Standard-Apps} fehlen.