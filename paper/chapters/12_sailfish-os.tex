Sailfish OS (auch \mbox{SailfishOS}) wurde 2011 von Jolla gestartet und erstmals Ende November 2013\thinspace\cite{online:jolla-about} veröffentlicht. Es wird hauptsächlich von Jolla entwickelt und basiert auf dem Linux Kernel, dem Mer Projekt welches Basisfunktionalitäten anbietet, einer eigenen UI, sowie standardmäßigen Apps für die nötigsten Funktionalitäten\thinspace\cite{online:sailfish-about}. Der Linux-Kernel ist GPL v2 lizenziert\thinspace\cite{online:kernel-license}, das Mer Projekt unter verschiedenen FOSS Lizenzen\thinspace\cite{online:mer-license}, die UI ist proprietär und die mitgelieferten Applikationen sind eine Mischung\thinspace\cite{online:sailfish-about} wobei eine private, nicht-kommerzielle Nutzung des Systems seitens Jolla erlaubt ist\thinspace\cite{online:sailfish-eula}. \mbox{SailfishOS} wird in 2 Versionen ausgeliefert, einer mit proprietären Komponenten und eine mit nur freier Software. Jolla liefert mit ihrer Hardware des weiteren noch die proprietären Erweiterungen\thinspace\cite{online:sailfish-about} ``Android runtime''\thinspace\cite{online:sailfish-android-runtime-licensor}, ``Exchange active sync''\thinspace\cite{online:microsoft-eas} und ``Text prediction engine''\thinspace\cite{online:jolla-xt9}\thinspace\cite{online:xt9-license} aus. Erstere emuliert eine Android-Umgebung welche es den Benutzern erlaubt Android Anwendungen mit gewissen Beschränkungen auszuführen\thinspace\cite{online:sailfish-android-runtime}. Letztere sind direkt oder indirekt mit Patenten versetzt.\\

\subsection{Das System}
Im Auslieferungszustand bekommen User die nötigsten Funktionalitäten vom System. Leider ist jedoch nur der Webbrowser und die Office-Suite unter einer freien Lizenz\thinspace\cite{online:sailfish-about}. Einen Kontrast zu diesen restriktiven Apps bildet das erlangen von ``root''-Rechten welches gut dokumentiert und schnell durchführbar ist\thinspace\cite{online:sailfish-root}.Die User-Interaktion erfolgt bei SailfishOS mittels Gesten, Buttons sucht man vergeblich. Der Grundgedanke dessen ist es eine Umgebung zu schaffen welche dem User ein schickes und elegantes Interface bietet welches den vollen Bildschirm ausnutzt\thinspace\cite{online:sailfish-ui}. Im Moment ist dies noch ein Grundbestandteil des Systems, es soll aber in Zukunft entfernt werden\thinspace\cite{online:sailfish-about}. Das System wird in diesem Fall freier. 

Applikationen welche nativ ausgeführt werden, können aus 2 App-Stores (``Jolla-Store'' und ``OpenRepos''\thinspace\cite{online:openrepos}) installiert werden, wobei das betreiben eines eigenen App-Stores technisch möglich ist. Die Applikationen werden im ``\mbox{.rpm}'' Format ausgeliefert, welches frei und offen\thinspace\cite{online:rpm-license} ist. Ein offenes Format erlaubt es den Nutzern zu überprüfen was das Paket am System bei der Installation verändern wird. Es stärkt hierbei also die Freiheit der Nutzer. Leider kann der Jolla-Store nicht ohne entsprechenden Clienten besucht werden und es ist zwingend ein Jolla-Account nötig\thinspace\cite{online:jolla-store}, Der Client für OpenRepos scheint nicht Open Source zu sein und der entsprechende App-Store hat weder eine Nutzungsbedingung, noch werden Lizenzen explizit erwähnt.

Anzumerken ist die Existenz von GCC im Auslieferungszustand. Dies ermöglicht Nutzern selbstgeschrieben Software direkt auf dem System zu kompilieren\thinspace\cite{online:sailfish-list-licenses} und auszuführen.

Applikationen welche emuliert ausgeführt werden (auf Android-Basis) sind aus den jeweiligen Quellen der Android Pakete und haben dessen entsprechende Lizenzierungen.\\

\subsection{Community}
Die Community setzt sich zu einem Großteil aus Technik-erfahreneren Leuten zusammen und jenen welche ein System benutzen wollen welches offener ist als die 3 aktuell größten Anbieter. Es gibt neben einer Mailing Liste noch ein Forum und mehrere IRC-Channels\thinspace\cite{online:sailfish-communitygeneral} in welchen diskutiert werden kann. Nebst diesen gibt es immer wieder Treffen in IRC-Channels bei welchen vorgeschlagene Themen angesprochen werden. Des weiteren nehmen SailfishOS Entwickler an Konferenzen teil und zeigen dort ihre Fortschritte oder anderweitig Neuerungen am System.\\
\iffalse
\subsubsection{Developer Support}
\thinspace\cite{online:jolla-store-terms}
https://harbour.jolla.com/\\
https://github.com/sailfishos\\
https://talk.maemo.org/showthread.php?t=92852\\
http://talk.maemo.org/showthread.php?p=1482661\#post1482661\\
https://github.com/maliit\\
https://talk.maemo.org/showthread.php?t=92036\\
\fi

\subsection{Hardware Support}
\mbox{SailfishOS} kann auf vielen verschiedenen Endgeräten installiert werden. Neben dem Raspberry Pi 2\thinspace\cite{online:sailfish-rpi2} wurde \mbox{SailfishOS} von der Community\thinspace\cite{online:sailfish-android-port} noch auf einige Android Geräte portiert. Diese Ports sind je nach Motivation und Anzahl der Entwickler mehr oder weniger ausgereift\thinspace\cite{online:sailfish-porters}.

Jolla ist bereit ihre Version an OEMs zu lizenzieren. Um Interesse zu wecken entwickelte Jolla Hardware --- Handy\thinspace\cite{online:jolla-smartphone} und Tablet\thinspace\cite{online:jolla-tablet} --- ist aber im Prozess der Restrukturierung in eine Software-Firma und einen Hardwarehersteller\thinspace\cite{online:jolla-googlelike}. Es lassen sich starke Parallelen zu Google ziehen und ihr Umgang mit Android sowie der Nexus-Linie.

Jolla's Plan trägt bereits erste Früchte. 2 Hardware-Hersteller haben sich gefunden welche \mbox{SailfishOS} auf ihren Geräten anbieten möchten. Fairphone auf ihrem Fairphone 2\thinspace\cite{online:jolla-fairphone2} und Intex Technologies (Zweit-größter Smartphone Hersteller indischen Ursprungs) wird Ende 2015 erste Hardware auf \mbox{SailfishOS} Basis in Indien anbieten\thinspace\cite{online:jolla-intex-pdf}. In beiden Fällen wird Jolla's proprietäre \mbox{SailfishOS} Version ausgeliefert.\\

\subsection{Fazit}
\mbox{SailfishOS} erfüllt mit seiner \mbox{``root''-Unterstützung}, dem rpm-Format und dem Nicht-Verlust der Garantie sowie Funktionalität unsere Aspekte. Jedoch ist es durch seine proprietären Komponenten bei den \mbox{Standard-Apps} und der \mbox{System-UI} leider nicht vollkommen offen und es muss der Entschluss gezogen werden, dass \mbox{SailfishOS} nicht vollkommen offen ist. Die FOSS-Version von \mbox{SailfishOS} ist leider nicht für den alltäglichen Gebrauch geeignet durch fehlende UI und Standard-Apps.