Einleitung (Developer, Jahrgang, Lizens, aktuell?, Hardware)
Sailfish OS (auch geschrieben SailfishOS), entwickelt hauptsächlich von Jolla, ist ein Betriebssystem auf Basis des Linux Kernels, dem Mer Projekt welches Basisfunktionalitäten anbietet, einer UI, sowie standardmässigen Apps für die nötigsten Funktionalitäten\cite{online:sailfish-about}. Der Linux-Kernel ist LGPLv2 lizensiert\cite{online:kernel-license}, das Mer Projekt unter verschiedenen FOSS Lizenzen\cite{online:mer-license}, die UI ist propriertär und die mitgelieferten Applikationen sind eine Mischung\cite{online:sailfish-about}\cite{online:sailfish-eula}. SailfishOS hat desweiteren noch in der Version von Jolla die proprietären Extras "Android runtime", "Exchange active sync" und "Text prediction engine". Erstere emuliert eine Android-Umgebung welche es den Benutzern erlaubt Android Anwendungen auszuführen (mit gewissen Beschränkungen\cite{online:sailfish-android}). Die beiden anderen Komponenten erlauben den Mailaustausch zu Microsoft Exchange Servern und noch eine Vereinfachung der Texteingabe auf dem Touchscreen. Jolla ist bereit ihr Betriebssystem an andere OEMs (Original Equipment Manufacturer) zu lizensieren, ähnlich wie Google es tut. Um aber Software gut entwickeln zu können entwickelt Jolla noch Hardware selbst. Dies ist in starker parallele zu der Nexus Serie von Google.\\

\subsection{Nackte System}
Im Auslieferungszustand bekommen User die nötigsten Funktionalitäten auf dem System. Auch ist es dem Benutzer ohne große Umstände möglich auf dem System ``root''-Rechte zu erhalten (https://together.jolla.com/question/30565/howto-using-su-instead-of-devel-su/). Mit diesen Rechten können Nutzer ohne größere Umstände auf dem System tiefgründige Änderungen vornehmen oder gar Apps installieren. Die User-Interaktion erfolgt bei SaifishOS mittels Gesten, Buttons sucht man vergeblich. Der Grundgedanke dessen ist es eine Umgebung zu schaffen welche dem User ein Interface bietet welches schickt und elegant ist und den vollen Bildschirm ausnutzt\cite{online:sailfish-ui}.
\subsubsection{Applikationen}
Applikationen welche nativ ausgeführt werden, können aus 2 App-Stores (``Jolla-Store'' und ``OpenRepos''\cite{online:openrepos}) installiert werden. Diese werden im ``{.}rpm'' Format ausgeliefert und installiert. Das format ist frei und offen\cite{online:rpm-license} für jeden, womit es auch Entwicklern einfach ist Packete anzubieten.\\
Applikationen welche emuliert ausgeführt werden (auf Android-Basis also) sind natürlich nicht diesem Format unterlegen und unterscheiden sich auch in der Lizens.\\

\subsection{Community}
\subsubsection{Developer Support}
https://harbour.jolla.com/

\subsection{Akuteller Stand}

\subsection{Legacy}
