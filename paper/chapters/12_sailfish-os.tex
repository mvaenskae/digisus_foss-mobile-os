Sailfish OS (auch \mbox{SailfishOS}) wurde 2011 von Jolla gestartet und erstmals Ende November 2013\cite{online:jolla-about} veröffentlicht. Es wird hauptsächlich entwickelt von Jolla und basiert auf dem Linux Kernel, dem Mer Projekt welches Basisfunktionalitäten anbietet, einer eigenen UI, sowie standardmäßigen Apps für die nötigsten Funktionalitäten\cite{online:sailfish-about}. Der Linux-Kernel ist LGPL v2 lizenziert\cite{online:kernel-license}, das Mer Projekt unter verschiedenen FOSS Lizenzen\cite{online:mer-license}, die UI ist proprietär und die mitgelieferten Applikationen sind eine Mischung\cite{online:sailfish-about} wobei eine private, nicht-kommerzielle Nutzung des Systems seitens Jolla erlaubt ist\cite{online:sailfish-eula}. \mbox{SailfishOS} hat des weiteren noch in der Version von Jolla die proprietären Erweiterungen ``Android runtime'', ``Exchange active sync'' und ``Text prediction engine''. Erstere emuliert eine Android-Umgebung welche es den Benutzern erlaubt Android Anwendungen auszuführen, jedoch mit gewissen Beschränkungen\cite{online:sailfish-android-runtime}. Zweitere, obschon offen dokumentiert ist nur unter Lizenz möglich mitzuliefern\cite{online:microsoft-eas} da es mit Patenten geschützt ist. Zu letzerem finden sich keine Informationen.\\
Jolla ist bereit ihr Betriebssystem an andere OEMs (Original Equipment Manufacturer) zu lizenzieren. Um genug Interesse zu wecken entwickelt Jolla Hardware --- Handy\cite{online:jolla-smartphone} und Tablet\cite{online:jolla-tablet} --- selbst. Es lassen sich starke Parallelen zu Google, ihrer Art mit Android umzugehen und der Nexus-Linie ziehen.\\

\subsection{Das System}
Im Auslieferungszustand bekommen User die nötigsten Funktionalitäten auf dem System. Auch ist es dem Benutzer ohne große Umstände möglich auf dem System ``root''-Rechte zu erhalten\cite{online:sailfish-root}. Mit diesen Rechten können Nutzer dann auf dem System tiefgründige Änderungen vornehmen. Die User-Interaktion erfolgt bei SailfishOS mittels Gesten, Buttons sucht man vergeblich. Der Grundgedanke dessen ist es eine Umgebung zu schaffen welche dem User ein Interface bietet welches schickt und elegant ist und den vollen Bildschirm ausnutzt\cite{online:sailfish-ui}. Im Moment ist dies noch ein Grundbestandteil des Systems, es soll aber in Zukunft entfernt werden\cite{online:sailfish-about}.
\subsubsection{Applikationen}
Applikationen welche nativ ausgeführt werden, können aus 2 App-Stores (``Jolla-Store'' und ``OpenRepos''\cite{online:openrepos}) installiert werden. Die Applikationen werden im ``\mbox{.rpm}'' Format ausgeliefert, welches frei und offen\cite{online:rpm-license} ist. Ein offenes Format erlaubt es den Nutzern zu überprüfen was das Paket am System bei der Installation verändern wird. Es stärkt hierbei also die Freiheit der Nutzer. Der Jolla-Store kann nicht ohne weiteres durchstöbert werden, es ist zwingend ein Jolla-Account und der entsprechende Client nötig\cite{online:jolla-store}, Der Client für OpenRepos lässt keinerlei Informationen auf sich ziehen und der entsprechende App-Store hat keinerlei Informationen zu angebotenen Apps schnell und ersichtlich sichtbar.\\
Mit dem OpenRepos-Store wird aufgezeigt, dass jeder seinen eigenen App-Store für \mbox{SailfishOS} betreiben kann. Dies ist jedoch nicht bei jedem System möglich und gehört deswegen extra erwähnt.\\
Applikationen welche emuliert ausgeführt werden (auf Android-Basis) sind aus den jeweiligen Quellen der Android Pakete und haben dessen entsprechende Lizenzierungen.\\

\subsection{Community}
Die Community setzt sich zu einem Großteil aus Technik-erfahreneren Leuten zusammen und jenen welche ein System benutzen wollen welches offener ist als die 3 aktuell größten Anbieter. Es gibt neben einer Mailing Liste noch ein Forum und mehrere IRC-Channels\cite{online:sailfish-communitygeneral} in welchen Ideen, Erfahrungen, Probleme, etc.\ angesprochen werden können. Das Forum ähnelt in seinem Aufbau dem von StackOverflow. Nebst diesen gibt es immer wieder Treffen in IRC-Channels bei welchen vorgeschlagene Themen angesprochen werden. Des weiteren nehmen SailfishOS Entwickler an Konferenzen teil und zeigen dort ihre Fortschritte oder anderweitig Neuerungen am System\cite{online:sailfish-communitygeneral}.
\subsubsection{Developer Support}
\iffalse
\cite{online:jolla-store-terms}
https://harbour.jolla.com/\\
https://github.com/sailfishos\\
https://talk.maemo.org/showthread.php?t=92852\\
http://talk.maemo.org/showthread.php?p=1482661\#post1482661\\
http://www.nuance.com/for-business/by-product/xt9/index.htm\\
https://github.com/maliit\\
https://talk.maemo.org/showthread.php?t=92036\\
\fi

\subsection{Hardware Support}
\mbox{SailfishOS} unterstützt noch weitere Hardware auf welcher das System genutzt werden kann. Neben dem Raspberry Pi 2\cite{online:sailfish-rpi2} wurde Sailfish noch auf eine Auswahl von Android Geräte von der Community\cite{online:sailfish-android-port} portiert. Diese Ports sind je nach Motivation und Anzahl der Entwickler mehr oder weniger ausgereift\cite{online:sailfish-porters}. Weitere Portierungen sind auf Nokia Internet Tablets (N8x0, N9x0, N9), das TI Pandaboard und das Raspberry Pi vollführt worden.\\
Es hat sich ebenfalls ein erster Hardware-Partner nach über 4 Jahren gefunden. Intex Technologies (Zweit-größter Smartphone Hersteller indischen Ursprungs), wird erste Hardware auf \mbox{SailfishOS} Basis in Indien Ende 2015 anbieten\cite{online:jolla-intex-pdf}.\\

\subsection{Fazit}
\mbox{SailfishOS} entspricht mit seiner ``root''-Unterstützung, dem rpm-Paket Format und dem Nicht-Verlust der Garantie sowie Funktionalität unseren Aspekten. Jedoch ist es durch seine proprietären Komponenten in Standard-Apps und System-UI leider nicht vollkommen offen und es muss der Entschluss gezogen werden, dass SailfishOS in seiner aktuellen Form nicht offen ist.