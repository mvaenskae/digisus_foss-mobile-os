\section{Android}

Android gehört heutzutage zu den bekanntesten un am weitesten verbeiteten mobilen Betriebssystemen. Es wird von Google entwickelt und basiert auf dem Linux Kernel. Die erste Version wurde im September 2008 veröffentlicht und mittlerweile ist es der grösste Konkurrent von Apples iOS. 

Da der Linuxkernel unter der GPL lizensiert ist, sind auch sämtliche Modifikationen des Kernels, die für Android vorgenommen wurden, so lizensiert. Für Software, die ausserhalb des Kernels läuft (sog. userspace software), wird die Apache Software License 2.0 verwendet \cite{online:android-licenses}. Die ASL 2.0 ist eine Freie Software Lizenz, die das Verändern und Weiterverteilen der Software erlaubt. Allerdings ist sie  keine copyleft-Lizenz, sie erlaubt es also, dass Modifikationen der Software anders lizenziert werden\cite{online:apache-license}.

\subsection{Rechtliche Offenheit}
Bei den Standard-Apps, die bei Android mitgeliefert werden, sieht es allerdings anders aus. Die meisten dieser 'Google-Apps' sind proprietär. Die Nutzungsbedingungen der Google Maps App beispielsweise verbieten es ausdrücklich, die Software weiterzuverteilen, zu dekompilieren oder zu 'reverse-engineeren'\cite{online:google-maps-tos}. Insbesondere dürfen die Hardware-Hersteller, die ihre Smartphones mit Android ausliefern möchten, nur entweder alle oder gar keine der Google Apps mit ausliefern. Sollten sie sich für die Google-Apps entscheiden, kommen noch einige weitere Restriktionen hinzu, beispielsweise muss Google Search als Standardsuchanbieter eingestellt sein\cite{online:mada-leak}.
Diese Restriktionen stammen aus dem 'Mobile Application Distribution Agreement', welches ein Hersteller, der Google-Apps auf seinen Geräten ausliefern will, unterzeichnen muss. Die Details dieses Dokuments sind nur dank eines Rechtsstreits zwischen Google und Oracle bekannt, während dem eine Version des 'Mobile Application Distribution Agreement' veröffentlicht wurde\cite{online:ars-mada-leak}. Leider ist diese Version vom Januar 2011 und somit schon wieder veraltet. Es steht anzuzweifeln, ob sich in der Zwischenzeit viel geändert hat. 

\subsection{Technische Offenheit}
Das Android-Betriebssystem verhindert standardmässig das Installieren von Software aus unbekannten Quellen, verfügt aber über eine Option in den Einstellungen, um dieses Verhalten auszuschalten. Auch das Erlangen von Root-Rechten ist machbar und resultiert, entgegen dem Glauben der meisten Nutzer, nicht in einem Garantieverlust\cite{online:xda-rooting-warranty}. Allerdings ist der notwendige Prozess für das Erhalten von Rootzugriff mit Risiken verbunden - um solch tiefgreifende Änderungen am System vornehmen zu können, muss in den meisten Fällen zuerst der Bootloader entsperrt werden. 
Ob und mit welchen Mitteln der Bootloader entsperrt werden kann, hängt vom Hardware-Hersteller ab\cite{online:apu-what-is-unlocking}. Nach unserer Definition ist der Bootloader nicht Teil des Betriebssystems, daher ist dieser Faktor für die Offenheit des Systems irrelevant.

\subsection{Fazit}
Google macht zwar viel Werbung mit Open Source Software und betont, dass diese eine zentrale Rolle spiele\cite{online:google-open-source}, und auch Android wird als 'offenes Betriebssystem' verkauft\cite{online:android-open-source}, doch wenn man genauer hinsieht bemerkt man, dass Kontrolle über den Markt höher gewichtet wird als die Free and Open Source-Ideologie. Technisch gesehen mag Android offen sein, doch Instrumente wie das 'Mobile Application Distribution Agreement' ermöglichen es Google, den Smartphone-Markt stark zu kontrollieren, indem wichtige Komponenten des Systems - die Apps - unter Verschluss gehalten werden. 
Nach unserer Definition ist Android also nur teilweise offen.
