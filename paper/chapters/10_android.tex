Android gehört heutzutage zu den bekanntesten und am weitesten verbeiteten mobilen Betriebssystemen. Es wird von Google entwickelt und basiert auf dem Linux Kernel. Die erste Version wurde im September 2008 veröffentlicht und mittlerweile ist es der grösste Konkurrent zu Apples iOS.
\newline

\subsection{Rechtliche Offenheit}
Der Linux Kernel ist GPL v2 lizenziert\thinspace\cite{online:kernel-license}, und folglich auch alle Modifikationen daran, die für Android vorgenommen wurden. Für Software, die ausserhalb des Kernels läuft (sog.\ userspace software), wird die Apache Software License 2.0 verwendet\thinspace\cite{online:android-licenses}. Die \mbox{ASL 2.0} ist eine Freie Software Lizenz, die das Verändern und Weiterverteilen der Software erlaubt. Allerdings ist sie  keine copyleft-Lizenz; sie erlaubt es also, dass Modifikationen der Software anders lizenziert werden\thinspace\cite{online:apache-license}.

Bei den mitgelieferten Standard-Apps sieht es anders aus. Die meisten dieser ``Google-Apps'' sind proprietär. Die Nutzungsbedingungen der Google Maps App beispielsweise verbieten es ausdrücklich, die Software weiterzuverteilen, zu dekompilieren oder zu ``reverse-engineeren''\thinspace\cite{online:google-maps-tos}. Insbesondere dürfen die \mbox{Hardware-Hersteller}, die ihre Smartphones mit Android ausliefern möchten, nur entweder alle oder gar keine der Google-Apps mit ausliefern. Sollten sie sich für die Google-Apps entscheiden, kommen noch einige weitere Restriktionen hinzu, beispielsweise muss ``Google Search'' als Standardsuchanbieter eingestellt sein\thinspace\cite{online:mada-leak}.

Diese Restriktionen stammen aus dem ``Mobile Application Distribution Agreement'', welches ein Hersteller, der Google-Apps auf seinen Geräten ausliefern will, unterzeichnen muss. Die Details dieses Dokuments sind nur dank eines Rechtsstreits zwischen Google und Oracle bekannt, während dem eine Version des ``Mobile Application Distribution Agreement'' veröffentlicht wurde\thinspace\cite{online:ars-mada-leak}. Leider ist diese Version vom Januar 2011 und somit schon wieder veraltet. Es steht anzuzweifeln, ob sich in der Zwischenzeit viel geändert hat.
\newline

\subsection{Technische Offenheit}
Das \mbox{Android-Betriebssystem} verhindert standardmässig das Installieren von Software aus unbekannten Quellen, verfügt aber über eine Option in den Einstellungen, um dieses Verhalten auszuschalten. Offiziell gibt es nur eine Quelle von Applikationen, den ``\mbox{Play Store}''; doch es besteht die Möglichkeit, \mbox{Software-Pakete} in Form von sogenannten ``\mbox{application packages}'' (``apk''s) direkt aus dem Internet herunterzuladen und zu installieren. Es gibt ausserdem alternative App Stores, beispielsweise ``\mbox{F-Droid}'', ein Katalog von Free and Open \mbox{Source-Apps}\thinspace\cite{online:f-droid}. Damit diese Alternativen funktionieren, muss aber zuvor Software aus unbekannten Quellen zugelassen werden.

Auch das Erlangen von \mbox{``root''-Rechten} ist machbar und resultiert, entgegen dem Glauben der meisten Nutzer, im Allgemeinen nicht in einem Garantieverlust\thinspace\cite{online:xda-rooting-warranty}. Allerdings ist der notwendige Prozess für das Erhalten von Rootzugriff mit Risiken verbunden --- um solch tiefgreifende Änderungen am System vornehmen zu können, muss in den meisten Fällen zuerst der Bootloader entsperrt werden. 
Ob und mit welchen Mitteln dies getan werden kann, hängt vom Hardware-Hersteller ab\thinspace\cite{online:apu-what-is-unlocking}. Nach unserer Definition ist der Bootloader nicht Teil des Betriebssystems, daher ist dieser Faktor für die Offenheit des Systems irrelevant.

Dennoch bleibt zu bemerken, dass das sogenannte ``rooten'' des Smartphone von den Entwicklern nicht vorgesehen ist; ansonsten gäbe es systemeigene Tools dafür.
\newline

\subsection{Community}
Zum Android Projekt gehören mehrere Mailinglisten, ein Forum und einige \mbox{IRC-Channels}\thinspace\cite{online:android-community}. Die Mailinglisten sind in Themen aufgeteilt und Entwickler, die mit der \mbox{Android-Plattform} arbeiten, sind eingeladen diese zu abonnieren\thinspace\cite{online:android-community}.

Es gibt aber auch Boards, die zwar nicht zum Android Projekt gehören, sich aber intensiv damit beschäftigen. Eines der bekanntesten ist \mbox{XDA Developers}\thinspace\cite{online:xda-developers}, eine Community, die sich zwar nicht ausschliesslich, aber zu einem grossen Teil mit Android beschäftigt. Das Board ist auch beliebt unter Entwicklern von ``ROMs'', also nicht offiziell unterstützten Android-Derivaten, und es finden sich dort von der Community entwickelte Ports von neuen Android-Versionen auf ältere Geräte.
\newline

\subsection{Fazit}
Google macht zwar viel Werbung mit Open Source Software und betont, dass diese eine zentrale Rolle spiele\thinspace\cite{online:google-open-source}, und auch Android wird als offenes Betriebssystem verkauft\thinspace\cite{online:android-open-source}, doch wenn man genauer hinsieht bemerkt man, dass Kontrolle über den Markt höher gewichtet wird als die ``Free and Open Source''-Ideologie. Technisch gesehen mag Android offen sein, doch Instrumente wie das ``Mobile Application Distribution Agreement'' ermöglichen es Google, den \mbox{Smartphone-Markt} stark zu kontrollieren, indem wichtige Komponenten des Systems --- die Apps --- unter Verschluss gehalten werden. 

Nach unserer Definition ist Android also nur teilweise offen.