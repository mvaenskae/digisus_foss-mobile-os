Firefox OS wird hauptsächlich von Mozilla entwickelt und basiert auf Gecko - dem HTML-Renderer von Firefox - und dem Linux Kernel\cite{online:ff-architecture}. Vorgestellt wurde es erstmals im Februar 2012\cite{online:ff-techhive-b2g}.

Der in Firefox OS verwendete Kernel wurde vom Android Open Source Project übernommen und ist daher GPL v2 lizenziert\cite{online:kernel-license}. Sämtliche von Mozilla entwickelte Software läuft unter der ``Mozilla Public License'' (MPL)\cite{online:mozilla-licensing}. Die MPL ist eine Open-Source-Lizenz und copyleft\cite{online:mpl}.

Andreas Gal, der damalige Director of Research von Mozilla, behauptete nach der Veröffentlichung von Firefox OS im 2012, dass im ganzen Betriebssystem keine proprietäre Software involviert sei\cite{online:knowyourmobile-b2g}. Leider blieb das nicht so: 2014 begann Mozilla, die ``W3C Encrypted Media Extension'' (EME) zu implementieren\cite{online:mozilla-eme}. Die EME  ist eine Erweiterung, die es ermöglicht, DRM-geschützte Webinhalte nativ - also ohne die Verwendung von Plugins - auszuführen. Dazu ist die Benutzung eines proprietären Entschlüsselungsmoduls nötig\cite{online:mozilla-eme}, welches mittlerweile in Gecko - und somit auch in Firefox OS - enthalten ist.

Grund für diesen Entscheid war, dass Firefox und Firefox OS, um mit der Konkurrenz durch Apple und Google mitzuhalten, das Abspielen von DRM-geschützten Inhalten ermöglichen muss. Netflix beispielsweise nutzt EME, um seine Inhalte auf Endgeräten abzuspielen. Firefox-Nutzer hätten einen anderen Browser verwenden müssen, um solche Inhalte abzuspielen\cite{online:mozilla-eme}.

Mozilla war über diesen Entscheid selbst nicht glücklick\cite{online:ff-drm-implementation}, hielt den Kompromiss aber für notwendig, um marktfähig zu bleiben. Das Resultat ist, dass Firefox OS nun kein vollständig freies Betriebssystem mehr ist. \\




