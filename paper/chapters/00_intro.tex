\subsection{Historisches}
Der Durchbruch von Smartphones auf dem internationalen Handymarkt hat zu einer technischen Revolution der Telefonindustrie geführt. Mit dem neuen Format entstand die Notwendigkeit, Systeme zu entwickeln, welche speziell für die damit verbundenen Anforderungen konzipiert wurden. Sei es die Optimierung der Laufzeit auf Batterie oder die neuen Interaktionsmöglichkeiten durch Touchscreens und eine Vielzahl von Sensoren, überall entstanden Entwicklungsbereiche, die davor nicht im Zentrum der Aufmerksamkeit gestanden hatten. Klassische Betriebssysteme waren diesen neuen Anforderungen nicht gewachsen, da sie für andere Zwecke entworfen worden waren. Es war an der Zeit, neue Systeme von Grund auf zu \mbox{entwickeln --- das} \mbox{Smartphone-Betriebssystem} war geboren.
Im Gegensatz zur klassischen \mbox{PC-Industrie} war der Aufstieg von Smartphones von Beginn an in der Hand einer geschlossenen Industrie. Insbesondere Apple prägte 2007 mit dem iPhone die Geschichte des Smartphones stark. Ein Jahr später stieg Google mit Android in den Markt. Microsoft begann 2009, ihr \mbox{PDA-Betriebssystem} Windows Mobile zum \mbox{Smartphone-OS} Windows Phone umzubauen. Nun, fast zehn Jahre später, wird der Markt noch immer von diesen drei Giganten dominiert. Wo bleiben die offenen (Free and Open Source Software) Systeme?
\newline

\subsection{Die 4 Freiheiten}
Freie Software bietet die folgenden vier Freiheiten\thinspace\cite{online:fsf_vier-freiheiten}:
\begin{itemize}
	\renewcommand\labelitemi{--}
	\item Die Freiheit, die Software für jeden Zweck auszuführen
	\item  Die Freiheit, ihren Quellcode einzusehen oder beliebig zu verändern
	\item Die Freiheit, Kopien davon weiterzugeben
	\item Die Freiheit, die Software zu verbessern und das Resultat zu veröffentlichen
\end{itemize}
\mbox{}

\subsection{Unsere Freiheiten}
Ein Freies Smartphone-Betriebssystem muss unserer Meinung nach neben diesen vier Freiheiten zusätzlich folgende Aspekte erfüllen:
\begin{itemize}
	\renewcommand\labelitemi{--}
	\item Das System muss vorsehen, dass die Benutzer Apps aus beliebigen Quellen ausführen können
	\item Benutzer müssen sämtliche technisch machbaren Rechte (\mbox{``root-Rechte''}) erlangen können, ohne dafür Exploits benutzen zu müssen
	\item Von Benutzern vorgenommene softwareseitige Änderungen dürfen weder Garantieverlust noch Funktionsverlust nach sich ziehen
\end{itemize}
Des Weiteren muss das System auf Hardware laufen, die offen genug ist, um obige Punkte zuzulassen.
\newline

\subsection{Definition Betriebssystem}
Von einem modernen Smartphone-Betriebssystem wird erwartet, dass es zusammen mit vorinstallierten Apps ausgeliefert wird, welche grundlegende Funktionalitäten bieten (z.B. Telefon, SMS, Kontakte, Mail, Kamera, Browser etc.). In diesem Bericht betrachten wir das Grundsystem und diese vorinstallierten Apps als Gesamtpaket und bezeichnen sie als ``Mobiles Betriebssystem“. Dies impliziert, dass in einem Freien Mobilen System auch die Standard-Apps frei sein müssen.
\newline

Im Folgenden betrachten wir sechs Smartphone-Betriebssysteme (Android, Firefox OS, Sailfish OS, Maemo, Ubuntu Phone sowie iOS) und analysieren diese im Bezug auf ihre technische sowie rechtliche Freiheit. Wir betrachten die jeweiligen Communities und vergleichen die Systeme untereinander.