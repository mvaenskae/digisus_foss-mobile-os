Ubuntu Touch basiert auf der beliebten gleichnamigen \mbox{Linux-Distribution} und wird seit 2011 von der Firma Canonical speziell für Smartphones und Tablets entwickelt\thinspace\cite{online:ubuntutouch-features}. Die Bedienung erfolgt dabei fast ausschliesslich über Wischbewegungen --- ausser Lautstärke und Sperrknopf gibt es keine Hardware-Buttons\thinspace\cite{online:ubuntutouch-blick}. Sogenannte Scopes zeigen Informationen aus einer oder mehreren Quellen an\thinspace\cite{online:ubuntutouch-scopes}. Das System ist noch sehr neu und nur teilweise fertig, doch sind die meisten üblichen Funktionalitäten bereits implementiert\thinspace\cite{online:ubuntutouch-blick}. Für Entwickler locken zwei Schienen der \mbox{App-Programmierung}: HTML5 für einfaches Design oder QML für native, effizientere Programmierung (letzteres basiert auf Qt und somit auf C++)\thinspace\cite{online:ubuntutouch-developers}. Apps laufen im Hintergrund ungehindert und können so die volle Rechenpower nutzen, was sich allerdings negativ auf die \mbox{Akku-Laufzeit} auswirkt\thinspace\cite{online:ubuntutouch-blick}.

Ubuntu Touch soll in Zukunft eine Symbiose mit der Desktop-Version eingehen und, angeschlossen an einen grösseren Monitor, gar einen vollständigen Computer darstellen\thinspace\cite{online:ubuntutouch-advantages},\thinspace\cite{online:ubuntutouch-edge}. Ob der bisherige Paketmanager apt und .deb-Pakete weiterhin dafür zum Einsatz kommen werden, ist fraglich - ein neuer, umstrittener Paketmanager ist in Entwicklung\thinspace\cite{online:ubuntutouch-snappy},\thinspace\cite{online:ubuntutouch-snappytalk}.

Canonical versucht schon seit längerem, Hardwarehersteller für Ubuntu Touch zu gewinnen. Ein Versuch, via Crowd-Funding \$32M für ein Telefon namens Ubuntu Edge zu sammeln\thinspace\cite{online:ubuntutouch-edge}, schlug 2013 mit einer Summe von weniger als \$13M fehl\thinspace\cite{online:ubuntutouch-edgefail}. Die App ``Ubuntu Touch for Android'' ermöglicht es nun, das System per \mbox{Dual-Boot} auf einem existierenden, gerooteten und hardwaretechnisch ausreichenden \mbox{Android-Gerät} zu installieren.
\newline

\subsection{Root-Rechte mit Nachteilen}
Als linuxbasiertes Betriebssystem müsste Ubuntu Touch eigentlich traumhaft sein, was die technische Offenheit betrifft. Leider ist dies nur begrenzt der Fall: zwar kann man die \mbox{Login-Daten} für root leicht im Internet finden --- im Auslieferungszustand ist das Wurzelverzeichnis aber als \mbox{nur-lesbares} Dateisystem eingehängt. Praktisch bedeutet das, dass selbst mit erlangten Root-Rechten nicht beliebige Änderungen vornehmbar sind, nicht einmal \mbox{apt-get} läuft\thinspace\cite{online:ubuntutouch-aptget}.

Canonical beschreibt, wie man das Wurzelverzeichnis in den \mbox{Lese-Schreib-Modus} schalten kann\thinspace\cite{online:ubuntutouch-readwrite}. Das System ist allerdings nicht dafür konzipiert und Sicherheit und Stabilität leiden mit jedem Eingriff. Dazu kommt, dass beim Zurückschalten auf \mbox{read-only} sämtliche Benutzerdaten verloren gehen.

Auch der Paketmanager ist fest in Canonicals Hand, Apps werden kontrolliert\thinspace\cite{online:ubuntutouch-publish}. Im Internet gibt es Versuche mit alternativen Stores, doch auch dafür müssen Mechanismen im System ausgehebelt werden\thinspace\cite{online:ubuntutouch-jailbreak}.

Dies ist nicht, was ich von einem technisch offenen Betriebssystem erwarte. Benutzer können nicht einfach Änderungen am System vornehmen und weiterverteilen, sondern sie müssen es erst ``knacken'' und verlieren dabei Komfort und Sicherheit. Das ist meines Erachtens nicht förderlich für eine gemeinsame, auf die Community gestützte Entwicklung, sondern ein Schritt zum von einem Grosskonzern verteilten und kontrollierten System, wie es zum Beispiel bei Google der Fall ist.
\newline

\subsection{Freie Software als falsche Fassade}
2013 hat Canonical die \mbox{FOSS-Community} mit der Aussage erschüttert, die Verteilung veränderter Versionen von Ubuntu sei nur mit der expliziten Erlaubnis von Canonical erlaubt, oder wenn sämtliche Firmenzeichen aus der Software entfernt würden\thinspace\cite{online:ubuntutouch-mjgIn}. Damit macht Canonical Ubuntu zu einem Trademark und schützt sich gegen dessen Weiterverbreitung. Dies sind schlechte Nachrichten für sämtliche \mbox{Ubuntu-Derivate}: die Entfernung der Firmenzeichen setzt das erneute Kompilieren sämtlicher Software voraus\thinspace\cite{online:ubuntutouch-mjgLicense} --- ein aufwändiger Prozess, der nicht ohne bedeutende Rechenkapazität durchführbar ist.

Mit diesen Massnahmen verstösst Canonical klar gegen die GPL\thinspace\cite{online:ubuntutouch-fsf}. Nach zwei Jahren Auseinandersetzungen mit der Free Software Community, insbesondere der FSF, wurde das Lizenzabkommen mit der Ausssage ergänzt, dass die Richtlinie die von Fremdlizenzen gewährten Rechte nicht einschränke und nur im legal möglichen Rahmen gelte\thinspace\cite{online:ubuntutouch-ip}. Dennoch ist die Rechtserklärung von Canonical an vielen Stellen unklar und lückenhaft formuliert\thinspace\cite{online:ubuntutouch-uncertain}. In der Folge kann sich ein Entwickler nicht mehr sicher sein, ob und wann er für die Verwendung von Ubuntu verklagt werden könnte. Das insofern unvermeidliche Zurateziehen professioneller Rechtsberatung ist für die Communities finanziell meist nicht machbar\thinspace\cite{online:ubuntutouch-mjgLicense}. In einem Gespräch mit Matthew Garrett\thinspace\cite{online:ubuntutouch-mjgConversation} erwähnte Mark Shuttleworth, der Kopf hinter Canonical, gar, dass Canonical nicht an einer Klarstellung der Sachlage oder Beantwortung von Fragen interessiert sei --- der Status Quo sei der Firma von Nutzen.

Mit ihrer neuen Politik bürdet Canonical Entwicklern eine grosse Last auf, die den Werten Freier Software zuwider kommt. Die Rechte ``Modify and Share'' werden klar erschwert\thinspace\cite{online:ubuntutouch-mjgLicense}. Von wahrhaft Freier Software kann von daher keine Rede mehr sein, weder für Ubuntu Desktop noch für Ubuntu Touch.
\newline

\subsection{Zu jung für eine nennenswerte Community}
Ubuntu Touch ist noch immer im Entwicklerstadium und es gibt noch fast keine Telefone, die von Haus aus mit dem System ausgeliefert werden\thinspace\cite{online:ubuntutouch-wikipedia}. Momentan experimentieren wohl noch hauptsächlich technisch interessierte und versierte Personen mit dem neuen Betriebssystem. Aufgrund der geplanten Vereinigung mit Ubuntu Desktop ist es meines Erachtens angebracht, die existierende Ubuntu Community in naher Zukunft zur Ubuntu Touch Community zu zählen.