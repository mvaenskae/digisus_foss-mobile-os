\subsection{Auf Hardwarenabieter}
Freie Mobile Betriebssysteme bieten Hardwareanbietern die Möglichkeit, existente Software mit wenig finanziellem Aufwand für ihre Zwecke verwenden zu können. Langwierige Lizenzabkommen und kostspielige juristische Abklärungen entfallen dabei vollständig oder beschränken sich auf ein Minimum, da die meisten Freien Lizenzen in einfachem Stil gehalten sind. Dabei ist der Hardwareanbieter frei, die Software beliebig anzupassen, zum Beispiel einen eigenen Flavour herauszugeben oder existierende Benutzerschnittstellen zu branden. Ein Beispiel für letzteres ist Android: Viele Smartphonehersteller haben davon ihre eigene Version, wie HTC Sense oder Samsung TouchWiz.

Allerdings zieht das Verwenden von Freier Software für einen Hersteller auch Konsequenzen mit sich: Oft wird FOSS von kleinen Gruppen unterhalten, die daran nach Lust und Laune schreiben --- die Qualität der Software hängt allzu oft von der Motivation und Tageslaune der freiwilligen Entwickler ab. Dieses Modell ist nicht marktfähig: Ein Hardwareanbieter muss ein Gerät ausliefern können, welches sowohl Hardware- als auch Softwaretechnisch stabil läuft. Kunden unterscheiden in diesem Sinne nicht zwischen Telefon und Betriebssystem sondern sehen ein einziges Gerät, auf das sie sich verlassen können wollen. Somit ist es am Hersteller, die verwendete Software zu unterhalten. Da diese von jemand anderem geschrieben wurden und eventuell keinen einheitlichen Konventionen entspricht, kann dies einen zusätzlichen Aufwand bedeuten. Dazu kommt, dass im Falle von Copyleft-Lizenzen der Hersteller die getane Arbeit ebenfalls frei bereitstellen muss, wodurch andere Firmen das Produkt einfacher replizieren können.

Die grossen Hardwareanbieter sind sehr vorsichtig, was technische Neuerungen anbelangt. Bereits im Markt etablierte Firmen haben keinerlei Interesse, ihre bewährten Positionen zu verlassen. Da die Branche sehr schnell voranschreitet und viel technisches Know-How zur Herstellung von zufriedenstellenden Produkten nötig ist, ist es für Einsteiger sehr schwierig, sich zu etablieren. Auch existiert in der Mobilbranche ein Sinn für Mode - die breite Masse kauft nicht das beste Gerät, sondern das modischste und beliebteste.

Dennoch bietet Freie Software Anreize für Hardwareanbieter. Gerade bei misstrauischen Kunden ist es ein strategischer Vorteil und ein Signal, den Quellcode der verwendeten Software offen zu legen.
\columnbreak

\subsection{Auf Endkunden}
Wieso sollte ein iPhone-Nutzer auf ein freieres mobiles Betriebssystem umsteigen?

Besonders technisch versierte Endkunden können von Freier Software stark profitieren, weil sie beliebige Änderungen vornehmen können. Sämtliche Vorteile Freier Software gelten auch für ein Freies mobiles Betriebssystem.

Auch Laien erhalten bei einem offenen Softwarekonzept mehr Auswahl, da es aufgrund der vereinfachten Situation mehr Entwickler und somit Diversität gibt. Dadurch erhalten sie die Möglichkeit, umzusteigen, wenn ihnen eine Entwicklung der Software nicht gefällt. Bei proprietärer, geschlossener Software (zum Beispiel beim iPhone) sind Kunden der Willkür des Herstellers vollkommen ausgeliefert --- wenn dieser beschliesst, dass etwas geändert wird, müssen alle Kunden folgen.

Allgemein ist Freie Software nachhaltiger, weil auf guten Ideen jederzeit von beliebigen Personen neue Konstrukte gebaut werden können. Damit können mit diesem Modell Ideen besser wiederverwendet und schnellere Fortschritte erzielt werden als im geschlossenen Raum.