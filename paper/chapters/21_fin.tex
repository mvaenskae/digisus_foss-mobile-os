Von den obigen Mobilen Betriebssystemen ist iOS ganz klar das geschlossenste und das rechtlich sowie technisch am stärksten  limitierte. Es ist jedoch schwierig zu entscheiden, welches der behandelten Systeme das freieste ist; keines ist vollständig offen. Ein völlig freies Smartphone ist eine Utopie: Grundlegend wichtige Technologie wie Baseband (GSM), mobile Daten etc. sind durch eine Vielzahl von Patenten geschützt. Auch ist es in den meisten Ländern nicht legal, alle benötigten Standards offen zu legen, weil Funksender unter staatlicher Kontrolle stehen.

Zum Schluss noch ein Gedankenspiel: Würde jeder Hersteller seine Software offen anbieten, könnte Android auf iPhones installiert werden und umgekehrt. Es wären Ports jeden Systems auf jede Plattform denkbar, und die Diversität wäre bedeutend höher. Würden dadurch alle Hersteller die Arbeit des jeweils anderen kopieren? Bereits heute, im geschlossenen Markt, gleichen sich die User Interfaces der verschiedenen Systeme immer mehr. Unserer Ansicht nach würde die Möglichkeit, ein System eins zu eins zu übernehmen, zu einer Gegentendenz führen: Um sich abzuheben, würden die Hersteller sich wohl mehr bemühen, wirklich Neues zu schaffen.