Maemo wurde von November 2005\thinspace\cite{online:maemo1-770} bis Oktober 2010\thinspace\cite{online:maemo5-n900} von Nokia entwickelt. Es gab 5 größere Releases und entsprechende Hardware seitens \mbox{Nokia}\thinspace\cite{online:maemo1-770}\thinspace\cite{online:n800-specs}\thinspace\cite{online:n810-specs}\thinspace\cite{online:n900-specs}. In diesem Bericht wird Maemo 5 untersucht, denn bei der Hardware vorheriger Generationen fehlt das GSM-Modul. Telefonieren war also mit vorherigen Iterationen nicht möglich.

Maemo 5 ist auf Debian aufgebaut\thinspace\cite{online:maemo-about} und ist demzufolge eigentlich ein Desktop Betriebssystem. Es hat folglich zu einem Großteil die gleichen Lizenzen wie die Komponenten, auf denen es aufbaut, aber es gibt auch einige geschlossene Komponenten\thinspace\cite{online:maemo5-components}. Diese wurden über die Jahre leider nicht offiziell ersetzt. Es war auch niemals der Plan von Nokia, Maemos Code ganz zu veröffentlichen und unter eine freie Lizenz zu stellen\thinspace\cite{online:maemo-slides}.
\newline

\subsection{Das System}
Maemo 5 erfüllt die von uns erwarteten Funktionalitäten im Bereich Standardapplikationen ohne größere Abstriche zu machen, jedoch sind sie durch fehlende Updates nicht mehr auf dem aktuellsten Stand, und einige Nokia-spezifische Dienste funktionieren nicht mehr, da Internetdienste fehlen. Im Großen und Ganzen erfüllen sie ihren Zweck dennoch. Die von uns erwarteten ``root''-Rechte können auf mehrere dokumentierte Wege erhalten werden und sind dazu noch integraler Bestandteil des Systems\thinspace\cite{online:maemo-root}. Diese Ausnahme im allseits geschlossenem mobilen Markt ist lobenswert.

Programme werden im ``.deb'' Format ausgeliefert und können von den Paketverwaltungen ``apt'' und ``dpkg'' installiert werden\thinspace\cite{online:maemo-packetinstalling}.
\mbox{maemo.org} bietet für die Releases eigene Paketquellen an, aus welchen Programme verschiedenen Stabilitätsgrades installiert werden können\thinspace\cite{online:maemo-extras}. Diese Paketquellen kann man mit den allseits bekannten \mbox{App-Stores} gleichsetzen, und sie erweitern lediglich das System. Ebenso können Pakete via Webbrowser angeschaut und von dort auch heruntergeladen  werden\thinspace\cite{online:maemo-store}\thinspace\cite{online:maemo-rawrepos}.

Des weiteren gibt es die CSSU\thinspace\cite{online:maemo-cssu} (Community Seamless Software Update), welches eine Quelle für Updates integraler Systemkomponenten ist. Bei der CSSU gibt es ebenfalls eine Aufteilung in stabile und instabile Pakete. Das letzte Update für die instabilen Pakete erfolgte am 11.\@ April 2015\thinspace\cite{online:maemo-cssuchangelog}. Demnach wird das System aktiv weiterentwickelt.

Ein besonderes Detail an Maemo ist, dass es dank seinem Vater Debian ebenfalls ein vollwertiges Linux-System ist. Es laufen also ein Großteil der üblichen Linux-Programme auch auf Maemo. Änderungen des Quellcodes sind nicht immer nötig.
\newline

\subsection{Community}
Die Community war von Anfang an integraler Bestandteil von Maemo. Nokia hat schon früh wichtige Features, beispielsweise den Desktop, unter öffentliche Lizenzen gestellt\thinspace\cite{online:maemo-hildon}. Diese Schritte haben zu Interesse seitens der Entwickler geführt und erlaubten es Maemo, in Zeiten von iOS und Android eine interessante Entwicklerumgebung zu werden. Selbstverständlichkeiten wie ein Portrait-Modus wurden von der Community nachentwickelt\thinspace\cite{online:maemo-portrait}. Des Weiteren sind tiefgreifende Systemänderungen wie das Anpassen des Assemblercodes\thinspace\cite{online:maemo-thumb}, Übertakten des Systems\thinspace\cite{online:maemo-overclocking}, detailliertere Videoaufnahmen\thinspace\cite{online:maemo-hdvideo} und Lesen von \mbox{USB-Medien}\thinspace\cite{online:maemo-usbhost}  möglich geworden dank der Offenheit des Systems und der Motivation einiger Entwickler diese in Maemo zu implementieren. Es ist ebenfalls möglich ein vollwertiges Debian in Maemo zu installieren und auszuführen\thinspace\cite{online:maemo-easydebian}\thinspace\cite{online:maemo-easydebianwiki}.

Selbst in der Forschung war das N900 beteiligt. Ein großer Erfolg war die Entwicklung einer neuen und freien API für die Kamera durch ein Team an der Stanford Universität in Kooperation mit dem Nokia Research Center Paolo Alto\thinspace\cite{online:maemo-fcam}. Damit wurde es möglich, das Kameramodul manuell zu steuern, zur damaligen Zeit noch unvorstellbar. Android und iOS haben mittlerweile ähnliche Funktionalitäten erhalten\thinspace\cite{online:maemo-fcamlegacy}.

Die schon angesprochene CSSU besteht aus reinem Community-Code und ist eines der stärksten Indizien dafür, dass Hardware alleine nicht die Lebensdauer einer Instanz von System bestimmt, sondern auch die Community von zentraler Bedeutung ist. Die CSSU arbeitet daran, geschlossene Komponenten durch Eigenentwicklungen zu ersetzen. Nokias Wunsch, die Community stark miteinzubeziehen, ist also ein voller Erfolg, selbst Jahre nach Übernahme der Handysparte durch Microsoft\thinspace\cite{online:nokia-microsoft}.
\newline

\subsection{Hardware Support}
Maemo wird offiziell nur auf den Nokia 770\thinspace\cite{online:maemo1-770}, Nokia N800\thinspace\cite{online:n800-specs}, Nokia N810\thinspace\cite{online:n810-specs} und N900\thinspace\cite{online:n900-specs} unterstützt. Es gibt aber noch ein weiteres Gerät, welches seit Jahren in einer kleinen Gruppe entwickelt wird. Das Neo900, der spirituelle Nachfolger des Nokia N900\thinspace\cite{online:maemo-neo900}, wird von Entwicklern der Maemo Community und des Openmoko Projektes selbstständig entwickelt\thinspace\cite{online:maemo-neo900team}. Es ist noch nicht klar, wie dieses Projekt Maemo beeinflussen wird.
\newline

\subsection{Fazit}
Maemo ist im Großen und Ganzen ein vollwertiges Linux-System auf mobiler Hardware. Es gibt dem User viele rechtliche Freiheiten, jedoch ist nicht jede Datei freigelegt; es sind teils proprietäre Komponenten   mit eingebaut. Nutzer können demnach nicht ihre vollen technischen Freiheiten genießen, jedoch erheblich mehr als bei vielen anderen Systemen.