Maemo wurde von November 2005\thinspace\cite{online:maemo1-770} bis Oktober 2010\thinspace\cite{online:maemo5-n900} von Nokia  entwickelt. Es gab 5 größere Releases und entsprechender Hardware Seitens Nokias\thinspace\cite{online:maemo1-770}\thinspace\cite{online:n800-specs}\thinspace\cite{online:n810-specs}\thinspace\cite{online:n900-specs}. In diesem Bericht wird Maemo 5 verstärkt angeschaut, denn vorherige Generationen wurden nicht mit Hardware ausgeliefert welche einem Smartphone gleichzusetzen waren. Es fehlte bei diesen das GSM-Modul.

Maemo 5 ist auf Debian aufgebaut\thinspace\cite{online:maemo-about} und ist demzufolge eigentlich ein Desktop Betriebssystem. Es hat also zu einem Großteil die gleichen Lizenzen wie die Komponenten auf denen es aufgebaut ist, aber es gibt auch einige geschlossene Komponenten\thinspace\cite{online:maemo5-components}. Diese sind über die Jahre leider nicht offiziel ersetzt worden und es war auch niemals Plan seitens Nokias Maemo's Code ganz zu veröffentlichen und unter eine freie Lizenz zu stellen\thinspace\cite{online:maemo-slides}.\\

\subsection{Das System}
Maemo 5 erfüllt die von uns erwarteten Funktionalitäten im Bereich Standardapplikationen ohne größere Abstriche zu machen, jedoch sind sie durch fehlende Updates nicht mehr auf dem aktuellstem Stand und einige Nokia-spezifische Dienste funktionieren nichtmehr durch fehlende Internetdienste. Im großen erfüllen sie ihren Zweck dennoch. Die uns erwarteten ``root''-Rechte können auf mehrere dokumentierte Wege erhalten werden und sind dazu noch integraler Bestandteil des Systems\thinspace\cite{online:maemo-root}. Dies ist eine Seltenheit und erhält ein großes Lob.

Programme werden im ``.deb'' Format ausgeliefert und können von den Paket Verwaltungen ``apt'' und ``dpkg'' installiert werden\thinspace\cite{online:maemo-packetinstalling}.
\mbox{maemo.org} bietet für die Releases eigene Paketquellen an aus welchen Programme verschiedenen Stabilitätsgrades installiert werden können\thinspace\cite{online:maemo-extras}. Diese Paketquellen kann man mit den allseits bekannten App-Stores gleichsetzen und erweitern lediglich das System. Ebenso können Pakete via Webbrowser angeschaut und von dort auch heruntergeladen  werden\thinspace\cite{online:maemo-store}\thinspace\cite{online:maemo-rawrepos}.

Des weiteren gibt es noch die CSSU\thinspace\cite{online:maemo-cssu} (Community Seamless Software Update) welches eine Quelle für Updates integraler Systemkomponenten ist. Bei der CSSU gibt es ebenfalls eine Aufteilung in stabile und instabile Pakete. Das letzte Update für die instabilen Pakete erfolgte am 11.\ April 2015\thinspace\cite{online:maemo-cssuchangelog}. Es wird also immer noch aktiv am System weiterentwickelt.

Eine besonderes Detail an Maemo ist, dass es Dank seinem Vater Debian ebenfalls ein vollwertiges Linux-System ist. Es laufen also ein Großteil der üblichen Linux-Programme auch auf Maemo. Quellcodeanderungen sind nicht immer nötig.\\ 

\subsection{Community}
Die Community ist von Anfang an integraler Bestandteil von Maemo gewesen. Nokia hat schon früh wichtige Features wie den Desktop unter öffentliche Lizenz gestellt\thinspace\cite{online:maemo-hildon}. Diese Schritte haben zu Interesse Seitens Entwickler geführt und erlaubten es Maemo eine interessante Entwicklerumgebung zu werden in Zeiten von iOS und Android. Selbstverständlichkeiten wie ein Portrait-Modus wurden nachgerüstet\thinspace\cite{online:maemo-portrait} und selbst tiefste Systemänderungen wie das Anpassen des Assemblercodes\thinspace\cite{online:maemo-thumb}, Übertakten des Systems\thinspace\cite{online:maemo-overclocking}, detailliertere Videoaufnahmen\thinspace\cite{online:maemo-hdvideo} und lesen von USB-Medien\thinspace\cite{online:maemo-usbhost}  waren möglich dank offenem System und dem Wunsch einiger Entwickler diese Ideen in Maemo zu implementieren. Es ist ebenfalls möglich ein vollwertiges Debian in Maemo zu installieren und auszuführen\thinspace\cite{online:maemo-easydebian}\thinspace\cite{online:maemo-easydebianwiki}.

Selbst in der Forschung war das N900 beteiligt. Ein großer Erfolg war eine neue API für die Kamera welches von einem Team an der Stanford Universität mit dem Nokia Research Center Paolo Alto entwickelt wurde und diese daraufhin veröffentlichte\thinspace\cite{online:maemo-fcam}. Damit wurde es möglich das Kameramodul manuell zu steuern. Zur damaligen Zeit eigentlich unvorstellbar. Android und iOS haben mittlerweile ähnliche Funktionalität erhalten\thinspace\cite{online:maemo-fcamlegacy}.

Die schon angesprochene CSSU besteht aus reinem Community-Code und ist eines der stärksten Indizien, dass Hardware alleine nicht die Lebensdauer einer Instanz von System bestimmt, sondern auch die Community von zentraler Bedeutung ist. Die CSSU arbeitet daran geschlossene Komponenten durch Eigenentwicklungen zu ersetzen. Nokias Wunsch die Community stark miteinzubeziehen ist also ein voller Erfolg, selbst Jahre nach Übernahme der Handysparte von Microsoft\thinspace\cite{online:nokia-microsoft}.\\

\subsection{Hardware Unterstützung}
Maemo wird offiziell nur auf den Nokia 770\thinspace\cite{online:maemo1-770}, Nokia N800\thinspace\cite{online:n800-specs}, Nokia N810\thinspace\cite{online:n810-specs} und N900\thinspace\cite{online:n900-specs} unterstützt. Es gibt aber noch ein weiteres Gerät welches seit Jahren in einer kleinen Gruppe entwickelt wird. Das Neo900, der spirituelle Nachfolger des Nokia N900\thinspace\cite{online:maemo-neo900}, wird von Entwicklern der Maemo Community und des Openmoko Projektes selbstständig entwickelt\thinspace\cite{online:maemo-neo900team}. Es ist noch nicht klar wie dieses Projekt Maemo beeinflussen wird.\\

\subsection{Fazit}
Maemo ist im großen und ganzen ein vollwertiges Linux-System auf mobiles Hardware. Es gibt dem User viele rechtliche Freiheiten, jedoch ist nicht jede Datei öffentlich gelegt sondern als proprietäre Komponente mit eingebaut. Nutzer können also nicht ihre vollen technischen Freiheiten genießen, jedoch erheblich mehr als bei vielen anderen Systemen.