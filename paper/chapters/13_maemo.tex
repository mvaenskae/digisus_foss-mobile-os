Maemo wurde von November 2005\cite{online:maemo1-770} bis Oktober 2010\cite{online:maemo5-n900} von Nokia als eine Alternative zum damaligen \mbox{SymbianOS} entwickelt. Es gab 5 größere Releases und entsprechender Hardware Seitens Nokias\cite{online:maemo1-770}\cite{online:n800-specs}\cite{online:n810-specs}\cite{online:n900-specs}. In diesem Bericht wird Maemo 5 verstärkt angeschaut, denn vorherige Generationen wurden nicht mit Hardware ausgeliefert welche einem Smartphone gleichzusetzen waren. Es fehlte bei diesen die Telefoniefunktion durch fehlendem GSM-Modul.

Maemo 5 ist auf Debian aufgebaut\cite{online:maemo-about} und ist demzufolge eigentlich ein Desktop Betriebssystem. Es hat also zu einem Großteil die gleichen Lizenzen wie die Komponenten auf denen es aufgebaut ist, aber es gibt auch einige geschlossene Komponenten\cite{online:maemo5-components}. Diese sind mit den Jahren leider nicht ersetzt worden.\\

\subsection{Das System}
Maemo 5 erfüllt die von uns erwarteten Funktionalitäten im Bereich Standardapplikationen ohne größere Abstriche zu machen, jedoch sind sie durch fehlende Updates nichtmehr auf dem aktuellstem Stand. Erfüllen tun sie ihren Zweck dennoch. TODO

Die Applikationen werden im ``.deb'' Format ausgeliefert.  

\subsection{Community}

\subsection{Hardware Unterstützung}

\subsection{Fazit}